\documentclass{beamer}

\usetheme{Boadilla} % or try Darmstadt, Madrid, Warsaw, ...
\definecolor{NYU}{rgb}{0.33984375, 0.0234375, 0.546875}
\usecolortheme[named=NYU]{structure} % or try albatross, beaver, crane, ...
\usefonttheme{default}  % or try serif, structurebold, ...
\setbeamertemplate{navigation symbols}{}
\setbeamertemplate{caption}[numbered]

\usepackage[english]{babel}
\usepackage[utf8]{inputenc}
\usepackage[T1]{fontenc}
\usepackage{parskip}
\usepackage{amsmath, amsthm, amssymb, amsfonts, mathtools, xfrac, dsfont}
\usepackage{hyperref}
\usepackage{bm}

% grouping and bookending
\newcommand{\pr}[1]{\left(#1\right)}
\newcommand{\br}[1]{\left[#1\right]}
\newcommand{\cbr}[1]{\left\{#1\right\}}
\newcommand{\floor}[1]{\left\lfloor#1\right\rfloor}
\newcommand{\ceil}[1]{\left\lceil#1\right\rceil}
\newcommand{\abs}[1]{\left|#1\right|}
\newcommand{\norm}[1]{\left\lVert#1\right\rVert}
\newcommand{\ip}[1]{\left\langle#1\right\rangle}
\renewcommand{\vec}[1]{\left\langle#1\right\rangle}
% derivatives
\newcommand{\der}[2]{\frac{d #1}{d #2}}
\newcommand{\mder}[2]{\frac{D #1}{D #2}}
\newcommand{\pder}[2]{\frac{\partial #1}{\partial #2}}
% common bold and script letters
\newcommand{\C}{\mathbb{C}}
\newcommand{\E}{\mathbb{E}}
\newcommand{\F}{\mathcal{F}}
\newcommand{\G}{\mathcal{G}}
\renewcommand{\L}{\mathscr{L}}
\newcommand{\N}{\mathbb{N}}
\renewcommand{\O}{\mathcal{O}}
\renewcommand{\P}{\mathbb{P}}
\newcommand{\Q}{\mathbb{Q}}
\newcommand{\R}{\mathbb{R}}
\renewcommand{\S}{\mathbb{S}}
\newcommand{\Z}{\mathbb{Z}}
% math operators
\DeclareMathOperator{\Cov}{Cov}
\DeclareMathOperator{\Var}{Var}
\let\Re\relax
\DeclareMathOperator{\Re}{Re}
\let\Im\relax
\DeclareMathOperator{\Im}{Im}
\DeclareMathOperator{\diag}{diag}
\DeclareMathOperator{\tr}{tr}
\DeclareMathOperator*{\argmax}{arg\,max}
\DeclareMathOperator*{\argmin}{arg\,min}
% misc
\newcommand{\mat}[1]{\begin{bmatrix}#1\end{bmatrix}}
\newcommand{\ind}[1]{\mathds{1}_{#1}}
\renewcommand{\epsilon}{\varepsilon}

\title[Parallel Multipole Methods]{Parallel Multipole Methods}
\institute[]{NYU Courant}
\author{Paul Beckman, Mariya Savinov}
\date{\today}

\begin{document}

\begin{frame}
  \titlepage
\end{frame}

% Uncomment these lines for an automatically generated outline.
%\begin{frame}{Outline}
%  \tableofcontents
%\end{frame}

\begin{frame}{Motivating Problem}
  \pause
  \begin{itemize}
  \item Consider a collection of interacting particles with a potential
  \pause
  \item \textbf{Total potential} at a point $x$ due to \textbf{particles} $x_j$ with \textbf{charges} $q_j$ is
  \[
  u(x) = \sum_{j=1}^n \dfrac{q_j}{\abs{x-x_j}} = \sum_{j=1}^n q_j \phi(x-x_j)
  \]
  \pause
  \item Separate sum into \textbf{near-field} and \textbf{far-field} contributions
  \pause
  \begin{itemize}
    \item Take \emph{aggregate} effect of far-field charges via \textbf{Taylor series expansion}
  \end{itemize}
  \end{itemize}
    \begin{figure}
    \begin{center}
    \includegraphics[width=0.85\textwidth, angle=0]{{far_field_image}.pdf}
    \end{center}
    %\caption{S}
    \label{fig:far_field_approx}
    \end{figure}
\end{frame}

\begin{frame}{Barnes-Hut Tree Code: Taylor Expansion}
  \pause
  \begin{itemize}
    \item For a set of points $x_j$, potential contribution can be approximated:
    \[
    \hspace{-0.5cm} \sum_{j\in\text{far-field}} q_j\phi(x-x_j) \approx \sum_{m=0}^p \underbrace{\br{\sum_{j\in\text{far-field}} q_j a_m(x_j-x^*)}}_{\text{weight }m} S_m(x^*-x)+ O(\delta^{p+1})
    \]
    \pause
    \item For $N$ particles, partition $[0,1]$ uniformly at $O(\log N)$ levels:
    \begin{figure}
      \begin{center}
      \includegraphics[width=0.65\textwidth, angle=0]{{binary_tree_FMM}.png}
      \end{center}
    \end{figure}
    \pause
    \item For each $T_{\ell, k}$ cell, compute weights $w_{\ell,k,m}$ $\longrightarrow$ total cost $O(N\log N)$
  \end{itemize}
\end{frame}


\begin{frame}{Tree Code by Barnes and Hut: Interaction List}
  \pause
  \begin{itemize}
  \item At each level, only 'interact' with up to 3 far-field cells \emph{not already included at higher levels}
  \pause
  \begin{figure}
  \begin{center}
  \includegraphics[width=0.55\textwidth, angle=0]{{interaction_list}.pdf}
  \end{center}
  %\caption{S}
  \label{fig:interaction_list}
  \end{figure}
  \pause
  \item Then for a point $x$, the potential is approximately
    \[
    u(x) = \sum_{\ell=1}^{O(\log N)}\sum_{m=0}^p w_{\ell,k(\ell),m} S_m\pr{x_{\ell, k(\ell)}^* - x} + O\pr{2^{-p}}
    \]
  \pause
  \item Total cost is $O(N\log N)$
  \end{itemize}

  \begin{figure}
    \begin{center}
    \includegraphics[width=0.65\textwidth, angle=0]{{varying_n_m32_p5}.png}
    \end{center}
    %\caption{S}
    \label{fig:serial_confirm}
    \end{figure}
\end{frame}

\begin{frame}{Parallelism}
    \pause
    \begin{itemize}
        \item Parallelizing with OpenMP
        \pause
        \begin{itemize}
            \item \emph{Challenge:}  \pause Recursive tree code may prevent significant speedup
        \end{itemize}
        \pause
        \item Two options of how to parallelize
        \pause
        \item \textbf{OPTION 1:} Parallelize the loops in the code
        \pause
        \begin{itemize}
            \item computing weights, adding near-field terms, and adding far-field terms 
            \pause
            \item \emph{Potential Problem:} Loops of varying sizes, possibly large overhead
        \pause
        \end{itemize}
        \item \textbf{OPTION 2:} Distribute the work in the tree among threads
        \pause
        \begin{itemize}
            \item At tree levels $\ell<O(\log r)$ for $r$ threads, open a new section
            \pause
            \item \emph{Potential Problem:} Nested parallelism 
        \end{itemize}
    \end{itemize}
\end{frame}

\begin{frame}{Results: Comparing Parallel Options}
    \pause
    \begin{figure}
      \begin{center}
      \includegraphics[width=0.85\textwidth, angle=0]{{varying_m_n32768_p5_4_8}.png}
      \end{center}
      %\caption{S}
      \label{fig:comp_vers}
    \end{figure}
    \texttt{crackle1}: Intel Xeon E5630 (2.53 GHz)
\end{frame}

\begin{frame}{Results: Strong Scaling}
    \pause
    $n=32768$, max points in box $2048$, and $p=5$
    \begin{figure}
    \begin{center}
    \includegraphics[width=0.5\textwidth, angle=0]{{strong_scalability}.png}
    \includegraphics[width=0.5\textwidth, angle=0]{{strong_scalability2}.png}
    \end{center}
    %\caption{S}
    \label{fig:strong_scalability}
    \end{figure}
    \texttt{crunchy1}: AMD Opteron 6272 (2.1 GHz) 
\end{frame}

\begin{frame}{Results: Weak Scaling}
    \pause
    $n=8192 \cdot r$, max points in box $=2048$, and $p=5$
    \begin{figure}
    \begin{center}
    \includegraphics[width=0.5\textwidth, angle=0]{{weak_scalability}.png}
    \includegraphics[width=0.5\textwidth, angle=0]{{weak_scalability2}.png}
    \end{center}
    %\caption{S}
    \label{fig:weak_scalability}
    \end{figure}
    \texttt{crunchy1}: AMD Opteron 6272 (2.1 GHz) 
\end{frame}

\begin{frame}{Supplemental Slides}
\end{frame}

\begin{frame}{Approximating far-field}
  \begin{itemize}
  \item Use \textbf{Taylor series expansion} of $\phi(x-x_j)$ for small $\delta = \dfrac{x_j-x^*}{x^*-x}$:
  \begin{align*}
  \phi(x_j-x) & = \phi\pr{(x^*-x)\pr{1+\delta}}= \phi(x^*-x)\phi(1+\delta)\\
  & \approx \phi(x^*-x)\br{\sum_{m=0}^p \dfrac{\phi^{(m)}(1)}{m!}\delta^m + O(\delta^{p+1})}\\
  & = \sum_{m=0}^p a_m(x_j-x^*)S_m(x^*-x) + O(\delta^{p+1})
  \end{align*}
  %$a_k(x_j-x^*) = \dfrac{\phi^{(k)}(1)}{k!}(x_j-x^*)^k$\quad and\quad $S_k(x^*-x)=\dfrac{\phi(x^*-x)}{(x^*-x)^k}$
  \item Then potential from $x_j\in\text{far-field}$ is
  \[
  \sum_{j\in\text{far-field}} q_j\phi(x-x_j) \approx \sum_{m=0}^p \br{\sum_{j\in\text{far-field}} q_j a_m(x_j-x^*)} S_m(x^*-x)+ O(\delta^{p+1})
  \]
  \item Accuracy depends on choice of $x^*$ and thus size of  $\delta = \dfrac{x_j-x^*}{x^*-x}$
  \end{itemize}
\end{frame}

\begin{frame}{Tree Algorithm by Barnes and Hut, complexity $O(N\log N)$}

  \begin{itemize}
    \item For $N$ particles, partition $[0,1]$ uniformly at $O(\log N)$ levels:

  \begin{figure}
  \begin{center}
  \includegraphics[width=0.65\textwidth, angle=0]{{binary_tree_FMM}.png}
  \end{center}
  %\caption{S}
  \end{figure}


    \item Let $T_{\ell, k}$ be the cell at level $\ell$ with index $k=1:2^{\ell}$ with center $x_{\ell,k}^*$.

    \item Compute weight at each cell, total cost $O(N\log N)$
    \[
    w_{\ell, k, m} = \sum_{x_j \in T_{\ell,k}} q_j a_m(x_j-x_{\ell,k}^*)
    \]

    \item For a point $x$: far-field components added at increasingly \emph{coarse} levels
    \[
    u(x) = \sum_{\ell=1}^{O(\log N)}\sum_{m=0}^p w_{\ell,k(\ell),m} S_m\pr{x_{\ell, k(\ell)}^* - x}
    \]
  \end{itemize}
\end{frame}


\end{document}